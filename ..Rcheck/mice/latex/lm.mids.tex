\HeaderA{lm.mids}{Linear Regression on Multiply Imputed Data}{lm.mids}
\keyword{misc}{lm.mids}
\begin{Description}\relax
Performs repeated linear regression on multiply imputed data set
\end{Description}
\begin{Usage}
\begin{verbatim}
lm.mids(formula, data, ...)
\end{verbatim}
\end{Usage}
\begin{Arguments}
\begin{ldescription}
\item[\code{formula}] a formula object, with the response on the left of a ~ operator, and the 
terms, separated by + operators, on the right.
\item[\code{data}] An object of type 'mids', which stands for 'multiply imputed data set', typically
created by function \code{mice()}.
\item[\code{...}] Additional parameters passed to \code{\LinkA{lm}{lm}}
\end{ldescription}
\end{Arguments}
\begin{Value}
An objects of class 'mira', which stands for 'multiply imputed repeated analysis'.
This object contains m lm.objects, plus some descriptive information.
\end{Value}
\begin{Author}\relax
Stef van Buuren, Karin Oudshoorn, 2000
\end{Author}
\begin{References}\relax
Van Buuren, S. \& Oudshoorn, C.G.M. (2000). Multivariate Imputation by Chained Equations: 
MICE V1.0 User's manual. Report PG/VGZ/00.038, TNO Prevention and Health, Leiden.
\end{References}
\begin{SeeAlso}\relax
\code{\LinkA{lm}{lm}}, \code{\LinkA{mids}{mids}}, \code{\LinkA{mira}{mira}}
\end{SeeAlso}
\begin{Examples}
\begin{ExampleCode}
data(nhanes)
imp <- mice(nhanes)     # do default multiple imputation on a numeric matrix
fit <- lm.mids(bmi~hyp+chl,data=imp)
\end{ExampleCode}
\end{Examples}

