\HeaderA{md.pattern}{Missing Data Pattern}{md.pattern}
\keyword{misc}{md.pattern}
\begin{Description}\relax
Display missing-data patterns.
\end{Description}
\begin{Usage}
\begin{verbatim}
md.pattern(x)
\end{verbatim}
\end{Usage}
\begin{Arguments}
\begin{ldescription}
\item[\code{x}] A data frame or a matrix containing the incomplete data. 
Missing values are coded as NA's. 
\end{ldescription}
\end{Arguments}
\begin{Details}\relax
This function is useful for investigating any structure of missing 
observation in the data. In specific case, the missing data pattern 
could be (nearly) monotone. Monotonicity can be used to simplify the 
imputation model. See Schafer (1997) for details. Also, the missing
pattern could suggest which variables could potentially be useful for
imputation of missing entries.
\end{Details}
\begin{Value}
A matrix with \code{ncol(x)+1} columns, in which each row corresponds to
a missing data pattern (1=observed, 0=missing). 
Rows and columns are sorted in increasing amounts of missing 
information. The last column and row contain row and column counts,
respectively.
\end{Value}
\begin{Author}\relax
Stef van Buuren, Karin Oudshoorn, 2000
\end{Author}
\begin{References}\relax
Schafer, J.L. (1997), Analysis of multivariate incomplete data. 
London: Chapman\&Hall.
\end{References}
\begin{Examples}
\begin{ExampleCode}
data(nhanes)
md.pattern(nhanes)
#     age hyp bmi chl    
#  13   1   1   1   1  0
#   1   1   1   0   1  1
#   3   1   1   1   0  1
#   1   1   0   0   1  2
#   7   1   0   0   0  3
#   0   8   9  10 27

\end{ExampleCode}
\end{Examples}

