\HeaderA{mice.impute.passive}{Elementary Imputation Method: Passive Imputation}{mice.impute.passive}
\keyword{misc}{mice.impute.passive}
\begin{Description}\relax
Derive a new variable based on the mice.imputed data
\end{Description}
\begin{Usage}
\begin{verbatim}
mice.impute.passive(data, func)
\end{verbatim}
\end{Usage}
\begin{Arguments}
\begin{ldescription}
\item[\code{data}] A data frame
\item[\code{func}] A formula specifying the transformations on data
\end{ldescription}
\end{Arguments}
\begin{Details}\relax
This is a special imputation function for so-called passive imputation.
Using this function, the user can specify, at any point in the mice 
Gibbs sampling algorithm, a function on the (mice.imputed) data. 
This is useful, for example, to compute a cubic version
of a variable, a transformation like $Q=W/H^2$ based on two variables, 
or a mean variable like $(x1+x2+x3)/3$. The so derived variables might be
used in other places in the imputation model.
The function allows to dynamically derive virtually any function 
of the mice.imputed data at virtually any time.
\end{Details}
\begin{Value}
\begin{ldescription}
\item[\code{t}] The tranformed data.
\end{ldescription}
\end{Value}
\begin{Author}\relax
Stef van Buuren, Karin Oudshoorn, 2000
\end{Author}
\begin{References}\relax
Van Buuren, S. \& Oudshoorn, C.G.M. (2000). Multivariate Imputation by Chained Equations: 
MICE V1.0 User's manual. Report PG/VGZ/00.038, TNO Prevention and Health, Leiden.
\end{References}
\begin{SeeAlso}\relax
\code{\LinkA{mice}{mice}}
\end{SeeAlso}

