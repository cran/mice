\HeaderA{mira}{Multiply Imputed Repeated Analysis}{mira}
\aliasA{print.mira}{mira}{print.mira}
\aliasA{summary.mira}{mira}{summary.mira}
\keyword{misc}{mira}
\begin{Description}\relax
The "mira" object is generated by the lm.mids and glm.mids functions.
The "mira" class of objects has methods for the following generic functions:
print, summary.
\end{Description}
\begin{Usage}
\begin{verbatim}
print.mira(x,...)
summary.mira(object, correlation, ...)
\end{verbatim}
\end{Usage}
\begin{Arguments}
\begin{ldescription}
\item[\code{x, object}] An object containing the m fit objects of a complete data analysis, 
plus some additional information.
\item[\code{correlation}] 
\item[\code{...}] not used
\end{ldescription}
\end{Arguments}
\begin{Value}
\begin{ldescription}
\item[\code{call}] The call that created the object.
\item[\code{call1}] The call that created the mids object that was used in 'call'.
\item[\code{nmis}] An array containing the number of missing observations per column.
\item[\code{analyses}] A list of m components containing the individual fit objects from each of the m complete data analyses.
\end{ldescription}
\end{Value}
\begin{Author}\relax
Stef van Buuren, Karin Oudshoorn, 2000
\end{Author}
\begin{References}\relax
Van Buuren, S. \& Oudshoorn, C.G.M. (2000). Multivariate Imputation by Chained Equations: 
MICE V1.0 User's manual. Report PG/VGZ/00.038, TNO Prevention and Health, Leiden.
\end{References}

